%% start of file `template.tex'.
%% Copyright 2006-2013 Xavier Danaux (xdanaux@gmail.com).
%
% This work may be distributed and/or modified under the
% conditions of the LaTeX Project Public License version 1.3c,
% available at http://www.latex-project.org/lppl/.


\documentclass[11pt,a4paper]{moderncv}
% possible options include font size ('10pt', '11pt' and '12pt'), paper size ('a4paper', 'letterpaper', 'a5paper', 'legalpaper', 'executivepaper' and 'landscape') and font family ('sans' and 'roman')

\moderncvstyle{banking}
% style options are 'casual' (default), 'classic', 'oldstyle' and 'banking'

\moderncvcolor{blue}
% color options 'blue' (default), 'orange', 'green', 'red', 'purple', 'grey' and 'black'

\renewcommand{\familydefault}{\rmdefault}
\renewcommand{\listitemsymbol}{$\diamond$ }
% to set the default font; use '\sfdefault' for the default sans serif font, '\rmdefault' for the default roman one, or any tex font name

%\nopagenumbers{}
% uncomment to suppress automatic page numbering for CVs longer than one page

%%%%%%%%%%%%%%%%%%%%%%%%%%%%%%%% character encoding
\usepackage[utf8]{inputenc}
% if you are not using xelatex ou lualatex, replace by the encoding you are using

%\usepackage{CJKutf8}
% if you need to use CJK to typeset your resume in Chinese, Japanese or Korean

%%%%%%%%%%%%%%%%%%%%%%%%%%%%%%%% adjust the page margins
\usepackage[scale=0.92]{geometry}

\setlength{\hintscolumnwidth}{2.5cm}
% if you want to change the width of the column with the dates

%\setlength{\makecvtitlenamewidth}{10cm}
% for the 'classic' style, if you want to force the width allocated to your name and avoid line breaks. be careful though, the length is normally calculated to avoid any overlap with your personal info; use this at your own typographical risks...

\usepackage{import}
\usepackage{graphicx}
\usepackage{ifthen}
\usepackage{enumitem}



\newif\ifen
\newif\ifpt
\newcommand{\en}[1]{\ifen#1\fi}
\newcommand{\pt}[1]{\ifpt#1\fi}

% relate to multicolumn
\usepackage{layout}
\usepackage{multicol}
\usepackage{lmodern}
\usepackage[ngerman]{babel}
\usepackage{amsmath}
% \usepackage[top=1in, bottom=1in, left=1in, right=1in]{geometry}
\firstname{}
\lastname{}

% \pttrue % Uncomment for portuguese document
\entrue % Uncomment for english document


%%%%%%%%%%%%%%%%%%%%%%%%%%%%%%%% personal data
\name{Tamiris}{Crepalde Martins}
%\title{Curriculum Vitae}
%\address{Estrada Retal dos Mil, nº44, Chapero, Seropédica, RJ, 23855-100}
%\vspace{9pt}
\phone[mobile]{+55 21 99632 9170}
%\phone[mobile]{024 99848 9386 [whatsapp]}
%\phone[fax]{+3~(456)~789~012}
\email{tamiriscrepaldemartins@gmail.com}
\homepage{https://br.linkedin.com/in/tamiriscrepalde}
\extrainfo{\href{https://github.com/tamiriscrepalde}{https://github.com/tamiriscrepalde}}


\begin{document}

\makecvtitle

\pt{
\section{Objetivo}
%\vspace{4pt}
\small{Engenheira de Machine Learning com quase quatro anos de experiência desenvolvendo projetos em Data Science buscando por um novo desafio como Cientista de Dados. Com visão sistêmica e habilidade de traduzir dados em insights de negócio para stakeholders, busco expandir meu conhecimento enquanto desenvolvo produtos de dados para auxiliar empresas a alcançarem seus objetivos de negócio.}


%\section{Interesses}
%\small{Sistemas de potência, planejamento, programação, fontes renováveis, geração distribuída, regulação, consultoria.}

\section{Habilidades}

\begin{itemize}
\item Python | Pandas | Matplotlib | Seaborn | Scikit learn | SQL | Git | Fortran
\item Google Cloud Platform | Flyte | Airflow | MLFlow | Metabase | VSCode | \LaTeX | MS Office
\item Inglês – Proeficiência profissional CECRL C1 | Espanhol - Comunicação básica (leitura e escuta)
\end{itemize}


\section{Experiência Profissional}
\vspace{4pt}
\begin{itemize}

	\item{\cventry{Nov20 - Atual}{Data Analyst Pleno}{Hurb}{Rio se Janeiro, RJ}{}{\vspace{2pt} Desenvolvimento de análises estratégicas e com destaque para um projeto de criação e implementação de um pipeline de envio de transações offline para o Google Analytics por meio de envios HTTP Request do Measurement Protocol. }}
	\subitem{\cventry{Mar20 - Nov20}{Data Analyst Júnior}{}{}{}{\vspace{2pt} Utilização de python para exploração, tratamento e análises de dados. Destaque no período, um projeto de tracking de dados utilizando Google Tag Manager e Google Analytics e implementação do Enhanced Ecommerce do Google. }}
	\subitem{\cventry{Ago19 - Mar20}{Data Science Trainee}{}{}{}{\vspace{2pt} Desenvolvimento de projetos diversos na área de análise de dados, exploração, tratamento e visualização de dados. Utilização de PostgreSQL, MySQL e Standard SQL para consulta e exploração de dados. Criação de visualizações por meio da plataforma Metabase. }}
	\vspace{6pt}

	\item{\cventry{Set16 - Ago18}{Estagiária Nível Superior}{CEPEL - Centro de Pesquisas de Energia Elétrica}{Rio se Janeiro, RJ}{}{\vspace{2pt} Fornecimento de suporte à equipe de desenvolvimento do Programa de Análise de Redes (ANAREDE) com utilização de linguagem Fortran. Meu principal projeto foi a implementação de um modelo de relatório de análise de contingências, por sugestão do Operador Nacional do Sistema, que se tornou meu projeto final de curso.}}
	\vspace{6pt}

	%\item{\cventry{Jul 16 - Jun 18}{Representante Discente}{Departamento de Engenharia Elétrica, UFRJ}{Rio se Janeiro, RJ}{}{\vspace{2pt} Atuei na busca de criação uma cultura de participação estudantil nas decisões departamentais, além de aproximar professores e alunos para solução de impasses diversos dentro do curso de Engenharia Elétrica e representa-los na COAA, Comissão de Orientação e Acompanhamento Acadêmico.}}
	\vspace{6pt}

	%\item{\cventry{Fev 14 - Jul 16}{Líder de Projeto | Capitã}{Equipe de Robótica UFRJ MinervaBots}{Rio se Janeiro, RJ}{}{\vspace{2pt} Responsável geral por uma equipe multidisciplinar de trinta e cinco alunos, fortaleci habilidades de trabalho em equipe, liderança, gestão de pessoas e projetos e \textit{design} em engenharia. Desenvolvi competências como modelagem e simulação com \textit{SolidWorks}, gerencimento através de metodologia \textit{Scrum}, gerenciamento através da plataforma Trello. Auxiliei na criação de processo seletivo com foco em avalição por competência, fui facilitadora de avaliação 360º, além de supervisionar os setores de gestão financeira e marketing.}}
	\vspace{6pt}

	%\item{\cventry{Jan 13-Abr 13}{Eletricista de Manutenção II}{CSN - Companhia Siderúrgica Nacional}{Volta Redonda - RJ}{}{\vspace{3pt}No período realizei acompanhamento da construção da planta de siderurgia Aços Longos na Usina Presidente Vargas, vivenciando metodologia 5S e método de gestão Ciclo PDCA. Foram desenvolvidos trabalhos referentes a manutenção elétrica geral como manutenção de quadros de distribuição de energia e reparos em instalações elétricas.}}
	%\vspace{6pt}

	%\item{\cventry{Junho 2010 - Julho 2011}{Balconista}{Cia do Livro}{Vassouras - RJ}{}{\vspace{3pt}No período foram realizados trabalhos envolvendo atendimento diferenciado ao cliente, controle e operação de caixa, controle e manutenção de estoque, organização e manutenção do espaço.}}
	%\vspace{6pt}

% 	\item{\cventry{Mar13 - Dez14}{Bolsista de iniciação científica}{LASUP - Laboratório de Aplicação de Supercondutores - Poli UFRJ}{Rio de Janeiro - RJ}{}{\vspace{3pt}Período de atividades no Laboratório de Aplicação de Supercondutores (LASUP) envolvendo realização de testes de estanqueidade nos criostatos utilizados pelo MAGLEV Cobra, trem de levitação magnética, entre outras atividades.}}

% 	\item{\cventry{Jul09 - Jan10}{Estagiário Nível Técnico}{CSN - Companhia Siderúrgica Nacional}{Volta Redonda - RJ}{}{\vspace{3pt}Programa de estágio desenvolvido na área de manutenção elétrica e mecânica de motores e geradores. O trabalho envolveu acompanhamento de inspeções e manutenções em toda a planta de aços planos da Usina Presidente Vargas, manutenção estática e estudo das estruturas que compõe os motores e geradores.}}

\end{itemize}

\section{Experiência Extra-Curricular}
\vspace{4pt}
\begin{itemize}

	\item{\cventry{Jul16 - Jun18}{Representante Discente}{Departamento de Engenharia Elétrica, UFRJ}{Rio se Janeiro, RJ}{}{\vspace{2pt} Atuei na busca de criação uma cultura de participação estudantil nas decisões departamentais, além de aproximar professores e alunos para solução de impasses diversos dentro do curso de Engenharia Elétrica e representa-los na COAA, Comissão de Orientação e Acompanhamento Acadêmico.}}
	\vspace{6pt}

	\item{\cventry{Fev14 - Jul16}{Líder de Projeto | Capitã}{Equipe de Robótica UFRJ MinervaBots}{Rio se Janeiro, RJ}{}{\vspace{2pt} Responsável geral por uma equipe multidisciplinar de trinta e cinco alunos, fortaleci habilidades de trabalho em equipe, liderança, gestão de pessoas e projetos e \textit{design} em engenharia. }}
	\vspace{6pt}
% 	Desenvolvi competências como modelagem e simulação com \textit{SolidWorks}, gerencimento através de metodologia \textit{Scrum}, gerenciamento através da plataforma Trello. Auxiliei na criação de processo seletivo com foco em avalição por competência, fui facilitadora de avaliação 360º, além de supervisionar os setores de gestão financeira e marketing.

\end{itemize}

\section{Qualificações Acadêmicas}
% \vspace{4pt}
\begin{itemize}

	\item{\cventry{Abr13 - Dez18}{Graduação}{Universidade Federal do Rio de Janeiro}{Rio de Janeiro - RJ}{\textit{Engenharia Elétrica}}{}}
	\vspace{6pt}

% 	\item{\cventry{Fev07 - Dez09}{Ensino Médio Técnico}{Escola técnica Pandiá Calógeras}{Volta Redonda - RJ}{\textit{Eletromecânica}}{}}

\end{itemize}


%\section{Experiência Profissional - Freelancer}
%\vspace{6pt}
%\begin{itemize}

	%\item{\cventry{2011 - 2013}{Recepcionista de Formaturas}{MP Eventos}{Volta Redonda e Região- RJ}{}{\vspace{3pt}Trabalhos pontuais envolvendo a realização de formaturas. As atribuições envolviam recepcionar os convidados, auxiliar equipe organizacional durante os eventos e liderar equipe de recepcionitas.}}
	%\vspace{6pt}

	%\item{\cventry{Junho 2010}{Garçonete}{Hotel Varandas}{Conservatória - RJ}{}{\vspace{3pt}Trabalhos pontuais como garçonete em um tradicional hotel fazenda em Conservatória. Atribuições envolvendo unicamente atendimento aos hospedes nos horários de refeição.}}
	%\vspace{6pt}

	%\item{\cventry{2010 - 2012}{Recepcionista em eventos universitários}{Diversos}{Volta Redonda - RJ}{}{\vspace{3pt}Estágio na área de manutenção elétrica de motores e geradores na GEO. Trabalho envolvendo manutenção deral de motores, tanto mecânica quanto elétrica em toda a planta da Usina Presidente Vargas.}}

%\end{itemize}


%\subsection{Projetos}
%\vspace{5pt}
%\begin{itemize}

	%\item{\textbf{Masters Project (Ongoing):} \textit{'Development of an Intelligent Humanoid Robot'}
	%\vspace{3pt}

%\end{itemize}


\section{HABILITAÇÕES E CERTIFICADOS}
%\vspace{2pt}
\begin{itemize}

	\item \textbf{Linguagens} Inglês \textit{avançado}; %Espanhol \textit{básico}.
	\vspace{4pt}

	\item \textbf{Programação} Fortran \textit{intermediário}; Python \textit{avançado}; SQL \textit{avançado}; MatLab \textit{básico}; Javascript \textit{básico}.
	\vspace{4pt}

	\item \textbf{Ferramentas} ANAREDE; Visual Studio; MS Office; Git; TortoiseSVN, Metabase; Google BigQuery; Airflow; Dataform.
	\vspace{4pt}

	\item \textbf{Cursos Extracurriculares} Python for Data Science and Machine Learning Bootcamp (Udemy - Atual); Exploring and Preparing your Data with BigQuery, Creating New BigQuery Datasets and Visualizing Insights, Achieving Advanced Insights with BigQuery, Applying Machine Learning to your Data with GCP (Google Cloud, 2020); The Complete SQL Bootcamp (Udemy, 9h, 2020); Curso Básico - ANAREDE (CEPEL, 21h, 2017); Curso FLUPOT (CEPEL, 21h, 2017).

\end{itemize}


% \section{Voluntariado}
% \vspace{4pt}
% \begin{itemize}

% 	\item{\cventry{Out10 - Atual}{TRE-RJ}{Voluntária para trabalho eleitoral}{Volta Redonda, RJ}{}{}}

% \end{itemize}


% \section{Extra-Curricular}
% \vspace{4pt}
% \begin{itemize}

% 	\item \textbf{Comissão de Formatura Engenharia Elétrica UFRJ 2018:} Atuei junto a uma equipe de outros quatro discente na organização da festa de formatura da turma de formandos do ano de 2018. Realizamos, além de pesquisas, reuniões e negociações com a empresa contratada visando atender as demandas da turma de vinte formandos.
% 	\vspace{6pt}

	%\item \textbf{Rede Alumni Núcleo, Fundação Estudar:} Rede alumni dos programas presenciais do Na Prática, da Fundação Estudar. Foco na criação de uma rede de jovens com vontade e potencial para transformar o Brasil, fornecendo experiências que estimulem o desenvolvimento pessoal e profissional dos membros. Seus pilares principais são liderança, redes, mercado e propósito.
	%\vspace{6pt}

% \end{itemize}


%\section{References}
%\vspace{6pt}
%\begin{itemize}

	%\item{Up to 4 references available on request}

%\end{itemize}
}



\en{
\vspace{-35}
% \section{Objective}

% \small{Mid-level Data Scientist with more than four years of experience developing projects in Data Science looking for a new challenge. With a systemic vision and the ability to translate business insights to stakeholders, I want to expand my knowledge while developing data products focusing on reaching business goals.}

\section{Professional Background}
\vspace{3}
\cventry{Aug 21 - Current}{Hurb}{Data Scientist}{Rio de Janeiro, RJ}{}{
	\cvlistitem{Developed a heuristic to predict hotel room prices by applying Kernel Density Estimation, which became the main resource to precify travel packages' hotel room nights.}
	\cvlistitem{Improved a Hotel Rooms Deduplication algorithm focusing on creating label functions based on business rules.}
	\cvlistitem{Developed a Demand Estimator using Google Vertex AI, Flyte as orchestrator, and MLFlow for monitoring. The project helped commercial executives to evaluate future demand and price to improve negotiations. It was an extensive data modeling challenge with plenty of business rules and constraints transformed into features.}
	\cvlistitem{Built a K-Means clusterization model to cluster hotel based on marketing performance metrics, helping Performance Marketing Team to improve bidding strategies. }
}
\vspace{6}

\cventry{Mar 20 - Aug 21}{Hurb}{Data Analyst}{Rio de Janeiro, RJ}{}{
	\cvlistitem{Developed a pipeline for offline transaction tracking through Google's HTTP Request Measurement Protocol, using Airflow as orchestrator, allowing the Performance Team to use accurate sales data to enhance Google Ads campaigns.}
	\cvlistitem{Implemented Google Analytics' Google Enhanced E-commerce using Google Tag Manager, enriching user interactions data collection and allowing the development of projects such as Image Selection and Custom Search Results based on CTR.}
	\cvlistitem{Worked giving support to the Commercial and Marketing Performance teams creating tables, views, and dashboards to support a wide variety of strategic decisions.}
}
\vspace{6}


\cventry{Aug 19 - Mar 20}{Hurb}{Data Science Trainee}{Rio de Janeiro, RJ}{}{
	% \cvlistitem{Supported stakeholders from teams like Product, Commercial, and Marketing. Labored extensively with Google Analytics data through SQL and BigQuery.}
	\cvlistitem{Supported business areas by using SQL to translate data into business insights through dashboardings and reports.}
	\cvlistitem{Developed a study case, using Google Trends data, about destinations with high demand for which we didn't use to sell travel packages; we could identify destinations that turned into top sellers in the following years.}
}


% \vspace{-10}
\section{Volunteering}
\vspace{3}
\cventry{Jul 23 - Current}{Omdena}{Data Scientist}{}{}{
	\cvlistitem{ [Togo Chapter] Development of a timeseries model to predict extreme weather events in Togo.}
	\cvlistitem{ [São Paulo Chapter] Development of a Coffee disease classifier to help Brazilian farmers addressing such problems.}
}

\vspace{-5}
\section{Skills}

\cvlanguage{Programming languages}{ Python, scikit-learn, SQL, Git, \LaTeX }{}
\cvlanguage{Other software/tools}{ GCP, BigQuery, VertexAI, Airflow, Flyte, Metabase, MLFlow, Machine Learnin,g }{}
% \vspace{-10}
\cvlanguage{}{Supervised learning, Unsupervised learning }{}
\cvlanguage{Interpersonal}{ Critical thinking, Problem solving, Analytical skills, Leadership }{}
\cvlanguage{Languages}{ Portuguese (Native); English (Professional proficiency CECRL C1)}{}


% \vspace{-10}
\section{Educational Background}

\cventry{Abr 13 - Dez 18}{Undergraduation}{Federal University of Rio de Janeiro}{Rio de Janeiro - RJ}{\textit{Electrical Engineering}}{}

% \cventry{Fev 07 - Dez 09}{Technical High School}{Pandiá Calógeras Technical School}{Volta Redonda - RJ}{\textit{Electromechanical}}{}


% \vspace{-15}
\section{Courses \& Certificates}

\cvitem{Data Scientist Nanodegree}{Udacity, \href{https://graduation.udacity.com/confirm/e/aac1149e-6961-11ec-8e17-e3a963d505e9}{\textcolor{beaublue}{certificate}}}{}{}{}{}
% \cventry{Mar 23}{International English Language Testing System}{British Council}{}{Band Score 7}{ }
\cvitem{Machine Learning Advanced Solutions Lab}{Google}{}{}{}{ }
\cvitem{From Data to Insights with Google Cloud Specialization}{Coursera, \href{https://www.coursera.org/account/accomplishments/specialization/certificate/QPCFQLLRBBQ2}{\textcolor{beaublue}{certificate}}}{}{}{}{ }

}
% end \en{}

\end{document}


%% end of file `template.tex'.
